\newcommand\tab[1][0,5cm]{\hspace*{#1}}

\chapter{Généralités}
\section{Introduction au sujet}
\paragraph{}
Ces dernières années, la communauté des lecteurs algériens a connu une croissance rapide et le taux de participation aux salons du livre est prometteur. Avec cette quantité de livres achetés, de nouveaux problèmes voient le jour, en plus des problèmes déjà existants tels que: les prix relativement élevés de certains livres, la rareté de certains autres et la tâche difficile que les lecteurs doivent endurer pour s'identifier avec d'autres membres de la communauté qui partage les mêmes intérêts et se localisant dans la même zone géographique, se pose le problème de stockage de ces livres et du fait qu’ils peuvent être mieux traités par un nouveau lecteur que de prendre de la place sur une étagère quelque part.
\paragraph{}
Il existe des clubs de lecture et des réunions sociales où les gens échangent, vendent, donnent des livres et même rencontrent de nouvelles personnes, ce qui laisse penser qu'il existe un groupe de personnes qui:
\begin{list}{•}{}
\item Ont des livres.
\item Sont disposés à abandonner les livres déjà lus.
\item Veulent économiser de l’argent tout en faisant lire des livres
\end{list}
\paragraph{}
Comme nous venons de le dire, certaines solutions existent pour cette communauté et nous en explorerons certaines dans les sections suivantes tout en soulignant leurs lacunes et en expliquant comment une fenêtre d'opportunités est encore ouverte pour résoudre le problème une fois pour toutes.
\newpage
\section{Solutions déjà existents}
\subsection{Non technologique}
\paragraph{}
L'analyse de cette question donnera lieu à deux concepts intéressants qui existent déjà et qui sont, dans une certaine mesure, fonctionnels:
\begin{enumerate}
\item \textbf{Événements d'échange de livres}\\
\tab Certains clubs de lecture et cafés organisent des événements où les participants sont invités à apporter les livres qu'ils souhaitent transmettre aux autres lecteurs et à en obtenir de nouveaux dans une atmosphère conviviale.
\subparagraph*{}
\begin{list}{•}{\textbf{Avantages}}
\item Les participants peuvent vivre toute l'expérience humaine d'échanger un livre.
\item Les participants sont à découvrir de nouveaux livres par des personnes autres que des simple avis en ligne.
\item Les participants établissent des liens significatifs avec leurs collègues lecteurs de livres.
\end{list}
\subparagraph{}
\begin{list}{•}{\textbf{Inconvénients}}
\item De tels événements durent au mieux 2 jours, ce qui fait que beaucoup de gens ratent l'occasion.
\item En raison de la limitation géographique, de tels événements ne peuvent être accessibles que par quelques résidents proches de la région.
\item coûteux en terme de temps investi.
\end{list}
\medskip \item \textbf{Échanges informels de livres}\\
\tab Certains échanges de livres sont informels - une étagère ou une boîte est fournie où les livres peuvent être laissés ou ramassés. L'échange repose sur les utilisateurs qui sortent et prennent des livres et n'est généralement pas supervisé.\\
\tab C'est une pratique courante dans les auberges de jeunesse où les voyageurs peuvent laisser un livre et emporter un livre différent avec eux. Certaines gares ferroviaires en Grande-Bretagne ont des échanges de livres informels et une a également été installée dans une cabine téléphonique à Kington Magna
\subparagraph*{}
\begin{list}{•}{\textbf{Avantages}}
\item Le processus est très rapide et pratiquement pas de temps perdu.
\item Absence de limite de temps, ils peuvent être échangés à tout moment.
\end{list}
\subparagraph*{}
\begin{list}{•}{\textbf{Inconvénients}}
\item Le besoin de donateurs au début du projet.
\item L'absence d'un système de contrôle de qualité pour les livres mis contre les prises.
\end{list}


\end{enumerate}
