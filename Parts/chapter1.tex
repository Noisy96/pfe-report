\chapter{Généralités}
\section{Introduction au sujet}
\paragraph{}
\bigskip Ces dernières années, la communauté des lecteurs algériens a connu une croissance rapide et le taux de participation aux salons du livre est prometteur. Avec cette quantité de livres achetés, de nouveaux problèmes voient le jour, en plus des problèmes déjà existants tels que: les prix relativement élevés de certains livres, la rareté de certains autres et la tâche difficile que les lecteurs doivent endurer pour s'identifier avec d'autres membres de la communauté qui partage les mêmes intérêts et se localisant dans la même zone géographique, se pose le problème de stockage de ces livres et du fait qu’ils peuvent être mieux traités par un nouveau lecteur que de prendre de la place sur une étagère quelque part.
\paragraph{}
Il existe des clubs de lecture et des réunions sociales où les gens échangent, vendent, donnent des livres et même rencontrent de nouvelles personnes, ce qui laisse penser qu'il existe un groupe de personnes qui:
\begin{list}{•}{}
\item Ont des livres.
\item Sont disposés à abandonner les livres déjà lus.
\item Veulent économiser de l’argent tout en faisant lire des livres
\end{list}
\paragraph{}
Comme nous venons de le dire, certaines solutions existent pour cette communauté et nous en explorerons certaines dans les sections suivantes tout en soulignant leurs lacunes et en expliquant comment une fenêtre d'opportunités est encore ouverte pour résoudre le problème une fois pour toutes.


