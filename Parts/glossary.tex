\makeglossaries
 
\newglossaryentry{freemium}
{
    name=freemium,
    description={un modèle commercial, en particulier sur Internet, dans lequel les services de base sont fournis gratuitement, tandis que les fonctionnalités les plus avancées doivent être payées}
}

\newglossaryentry{Webhook}
{
    name=Webhook,
    description={Un Webhook en développement Web est une méthode permettant d’augmenter ou de modifier le comportement d’une page Web, ou d’une application Web, avec des rappels personnalisés. Ces rappels peuvent être gérés, modifiés et gérés par des utilisateurs tiers et des développeurs qui ne sont pas nécessairement affiliés au site Web ou à l'application d'origine. Le terme "webhook" a été inventé par Jeff Lindsay en 2007 à partir du terme de programmation pour ordinateur\cite{noauthor_webhook_2019}}
}

\newglossaryentry{GPS}
{
    name=GPS,
    description={Un Global Positioning System (GPS) est un système de géolocalisation par satellite}
}

\newglossaryentry{coordGPS}
{
    name=coordonnées,
    description={Les coordonnées géographiques permettent de localiser un lieu sur la Terre grâce à trois mesures : l'altitude, la longitude et la latitude. Les coordonnées géographiques sont notamment utilisées par les GPS}
}

\newglossaryentry{OpenStreetMap}
{
    name=OpenStreetMap,
    description={OpenStreetMap est un projet international fondé en 2004 dans le but de créer une carte libre du monde. Il collecte des données dans le monde entier sur les routes, voies ferrées, les rivières, les forêts, les bâtiments …}
}

\newglossaryentry{BETA}
{
	name=BETA,
	description={Une pré-version du logiciel est donnée à un grand groupe d’utilisateurs à essayer dans des conditions réelles. Les versions bêta ont subi des tests alpha en interne et sont généralement assez proches du point de vue de la présentation et de la fonctionnalité du produit final; cependant, des modifications de conception sont souvent apportées.\cite{noauthor_beta_nodate}}
}

\newglossaryentry{GoogleMaps}
{
    name=Google Maps,
    description={Google Maps est une application pour la manipulation de données géographiques, permettant entre autre, d'afficher les cartes du monde entier et de tracer des itinéraires entre deux points par exemple}
}

\newglossaryentry{plateforme}
{
    name=plateforme,
    description={une plateforme est un environnement permettant la gestion et/ou l'utilisation de logiciels et applications, elle peut désigner un système d'exploitation, un environnement d'exécution, un serveur...etc}
}

\newglossaryentry{SQL}
{
    name=SQL,
    description={SQL (Structured Query Language) est un langage informatique normalisé servant à exploiter des bases de données relationnelles}
}

\newglossaryentry{NoSQL}
{
    name=NoSQL,
    description={NoSQL désigne une famille SGBD qui s'écarte du paradigme classique des bases de données relationnelles}
}

\newglossaryentry{framework}
{
    name=framework,
    description={Un framework est une sorte d'infrastructure de développement, il désigne un ensemble cohérent de composants logiciels structurels qui sert à créer les fondations ainsi que les grandes lignes de tout ou d'une partie d'un logiciel ou application}
}

\newglossaryentry{librairie}
{
    name=bibliothèque,
    description={En informatique, une bibliothèque logicielle est une collection de fonctions, qui peuvent être déjà compilées et prêtes à être utilisées par des programmes}
}

\newglossaryentry{API}
{
    name=API,
    description={une API (Application Programming Interface) est un ensemble de règles et spécifications qu'un programme peut suivre pour accéder et faire usage des services et des ressources fournis par un autre programme particulier qui implémente cette API}
}

\newglossaryentry{JSONg}
{
    name=JSON,
    description={JavaScript Object Notation}
}