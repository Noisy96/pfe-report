\newpage

\chapter*{Conclusion \& perspective}

\addcontentsline{toc}{chapter}{Conclusion \& perspective}
Le travail présenté dans ce mémoire licence décrit les outils et les étapes, ainsi que la base de connaissances nécessaire à la création d'une application mobile, qui est une plate-forme qui sert à faciliter le processus d'échange de livres entre lecteurs en fonction de leur localisation géographique.\medskip

Tout au long du parcours visant à donner vie à ce projet, et pour le fait que c’était un projet conjoint entre deux étudiants. quelques leçons importantes ont été apprises, Tout d’abord, je crois que maintenant nous réalisons tous les deux vraiment la valeur de publier un document détaillé et très spécifique. À la fin de chaque étape du processus, ainsi que la communication entre les créateurs du projet étant très crucial., et ça sans parler des connaissances techniques et de l'expérience véritablement acquises, de la conception de l'interface utilisateur à l'utilisation efficace de la documentation logicielle en ligne.\medskip

Ce travail n’est pas un projet achevé, et nous n’avions pas l’intention de le rendrait. Nous voulions que ce projet soit une excellente expérience d’apprentissage qui, d’une manière primordiale, devrait remettre en défi notre zone de confort et nos compétences. À des fins particulières, un cadre nouvellement créé a été utilisé, et par un langage de programmation qui, avant ce projet, aucun de nous ne le connaissait. Et parce que nous manquions volontairement d'expérience et de connaissances sur la technologie que nous utilisions, nous avons adopté une stratégie qui s'est révélée efficace dans nos circonstances actuelles. Elle consistait à créer d'abord la partie frontale de l'application, indépendamment du \gls{back-end} avec des données factices (\gls{dummy data}), puis travaillez progressivement à remplacer ces fausses données par des données réelles extraites des serveurs.\medskip

Cette première version de l'application peut servir autant qu'un \acrshort{MVP} (Minimum viable product) très utile. En utilisant les outils d'analyse de données fournis par Firebase, nous pouvons facilement étudier le comportement de nos utilisateurs et émettre des hypothèses sur la manière d'optimiser davantage notre expérience utilisateur tout en ajoutant de nouvelles fonctionnalités et en élargissions ce que ce produit peut réaliser. Cependant, avant même d’obtenir les commentaires de nos utilisateurs, nous étions encore en mesure de formuler des hypothèses sur ce qui pouvait être ajouté à l’application et ce qui pouvait être amélioré pour le mieux:

\begin{description}

\item[• Images réelles:] pour le moment, et afin de garder une interface utilisateur propre, nous avons choisi de ne pas donner la liberté d'ajouter de nombreuses images d'un livre. Nous avons insisté pour que les utilisateurs utilisent des couvertures numérisées plutôt que photographiées. Je pense que cette limitation que nous avons dû imposer aux utilisateurs ne durera pas longtemps, car l'expérience utilisateur sera de plus en plus personnalisable.

\item[• Suivi des transactions:] nous supposons que la mise en place d'un mécanisme de suivi des transactions peut être très bénéfique pour les utilisateurs, en particulier ceux qui prêtent plusieurs livres et aiment savoir où se trouvent leurs livres et quand ils ont été prêtés.

\item[• Découverte de livres:] pour les aventuriers qui ne savent pas exactement ce qu’ils recherchent, nous pouvons ajouter différentes méthodes pour trouver des livres, par exemple: présenter divers livres appartenant à des personnes géographiquement proches et les présenter un par un, où l'utilisateur voit les détails ou ignore chaque option.

\item[• Suggestion de livres:] compte tenu de la quantité de données que nous pouvons obtenir de la liste de livres de chaque utilisateur et des requêtes de recherche qu’il entre, nous pouvons implémenter une fonction de suggestion sous forme de notification push concernant les livres qu’un utilisateur pourrait trouver intéressants.

\item[• Une meilleure architecture logicielle:] au fur et à mesure que l'application évoluera, le maintien et la modification des fonctionnalités de l'application s'avéreront fastidieux. La prochaine décision rationnelle consistera donc à mieux structurer l'application de manière à permettre un processus de dimensionnement en douceur.

\end{description}
